\documentclass{article}

\usepackage{bussproofs}
\usepackage{amsmath}

\title{\vspace{-2cm}Popr Typing Rules}
\author{Dustin DeWeese}
\date{August 17, 2017}

\begin{document}

\maketitle

\section{Preliminaries}

Every term in Popr is a function from an input tuple to an output tuple, expressed as:

$$ a_0 \ldots a_{n-1} \to b_0 \ldots b_{m-1} $$

\noindent where lower case letters represent type variables. \\

Uppercase letters represent a group of types variables, so this can also be written as:

$$ A \to B $$

When unifying group type variables, they should match only the minimum number of elements required (non-greedy).

\section{Rules}

The following rules describe function composition (represented by adjacent terms). There are two cases, for when the left function produces a larger tuple than the right function consumes, and when the right function consumes more than the left function produces. Either case can be used if the tuple sizes are matched.

\def\proofSkipAmount{\vskip 1.5ex}

\begin{prooftree}
  \AxiomC{$ f : A \to B\ D $}
  \AxiomC{$ g : D \to C $}
  \RightLabel{\scriptsize $ com_L $}
  \BinaryInfC{$ f g : A \to B\ C $}
\end{prooftree}

\begin{prooftree}
  \AxiomC{$ f : A \to D $}
  \AxiomC{$ g : D\ B \to C $}
  \RightLabel{\scriptsize $ com_R $}
  \BinaryInfC{$ f g : A\ B \to C $}
\end{prooftree}

\vspace{5mm}

A quote contains a function:

\begin{prooftree}
  \AxiomC{$ f : A \to B $}
  \RightLabel{\scriptsize $ quote $}
  \UnaryInfC{$ [f] : [A \to B] $}
\end{prooftree}

Alternates are the compositional equivalent of logical disjunction:

\begin{prooftree}
  \AxiomC{$ f : A \to (B \big| C) $}
  \AxiomC{$ \Gamma, f : A \to B \vdash g : A \to D $}
  \AxiomC{$ \Gamma, f : A \to C \vdash g : A \to D $}
  \RightLabel{\scriptsize $ altE $}
  \TrinaryInfC{$ g : A \to D $}
\end{prooftree}

\begin{prooftree}
  \AxiomC{$ f : A \to B $}
  \RightLabel{\scriptsize $ altI_L $}
  \UnaryInfC{$ f : A \to (B \big| C) $}
\end{prooftree}

\begin{prooftree}
  \AxiomC{$ f : A \to C $}
  \RightLabel{\scriptsize $ altI_R $}
  \UnaryInfC{$ f : A \to (B \big| C) $}
\end{prooftree}

\section{Axioms}

\subsection{Quotes}

These functions allow manipulating quotes by composing them, supplying inputs, and extracting outputs.

\begin{align*}
\circ_L &: [A \to B\ D]\ [D \to C] \to [A \to B\ C] \\
\circ_R &: [A \to D]\ [D\ B \to C] \to [A\ B \to C] \\
pushl &: a\ [B \to C] \to [a\ B \to C] \\
popr &: [A \to B\ c] \to [A \to B]\ c \\
\end{align*}

\subsection{Branching}

Branching in Popr is nondeterministic. There is an operator to split a branch $(\big|)$, and an operator to conditionally prune a branch $(!)$.

\begin{align*}
\big| &: a\ b \to (a \big| b) & (alt) \\
! &: a\ True \to a & (assert) \\
\end{align*}

\subsection{Shuffling}

Basic ``stack'' shuffling operators.

\begin{align*}
swap &: a\ b \to b\ a \\
dup &: a \to a\ a \\
drop &: a\ b \to a \\
\end{align*}

\subsection{Types}

Supported concrete types are integers, quotes, and \textit{symbols}, which are a set of unique values. Notably, functions are \textbf{not} first class values. Because functions are not first class values, they also can not be curried.

Because symbols can only be compared for equality, every symbol type is actually a subset of all symbols. For example, $True$ and $False$ are two symbols used to represent a boolean value.

\subsection{Literals}

Literals are functions that append a value to the input tuple.

\begin{align*}
  \mathbf{Integer} &: A \to A\ Int \\
  \mathbf{Symbol} &: A \to A\ \mathbf{Symbol} \\
  [] &: A \to A\ [B \to B] \\
\end{align*}

\subsection{Other Primitives}

Several integer and comparison operators are implemented with the expected semantics. Comparisons output $True$ or $False$.

\end{document}